%%%%%%
%	Præamble
%%%%%%

\documentclass{article}

\include{./praeambel}

%%%%%%
%	Indhold
%%%%%%

\begin{document}

\section*{Warhammer Fantasy Roleplay Character Manager}
Rollespil er min store passion og har været det over 20 år, men et stående problemer er, at der kan være en en del papirarbejde at holde styr på.\newline{}
Det er selvfølgelig ikke ens for alle system og et system, hvor mine spillere konstant har bedt mig regne deres karakter efter er \textit{Warhammer Fantasy Roleplay}. I startet med at være fokusere meget på xp udregning, men endte med at blive nedprioriteret til et \textit{MVP}, som er bare en \textit{"character manager"}.

\section*{Produkt}
Produktet er en \textit{Blazor Web App} skrevet i \textit{Core 8.0}.\newline{}
Der er bygget en backend API med 7 modeller, 1 interface og 7 controllers.\newline{}
\textbf{Modellerne} og \textbf{interface}'n er bygger fra bunden\newline{}
\textbf{Controller}'erne er genere\textit{Entity Framework} og justerede og udbygget efter behov.\par{}

På Clientsiden giver \textit{Characters.razor} og \textit{Parties.razor} et overblik over de indtastede medlemmer af henholdsvis \textbf{Character} og \textbf{Party} klasserne.\newline{}
\textit{CharacterSheet.razor} udgør \textbf{edit} funktionen til \textbf{Character}-instanserne og er ikke færdig. Flere af implementeringer afhang dog af den side, så derfor er 4 af \textit{controller}'ne/\textit{modelerne} ikke rigtigt implementeret, men jeg har bibeholdt dem for at man kan se arbejdet der foreligger.\par{}

I forhold til \textit{API} er det nok værd særligt kigge \textbf{Character}-klasse og dens initialisering af \textbf{Characteristic}-klasen, som er gjort via API-kald. I begge \textbf{Skill} klasser arbejder jeg med tilkobling på en anden måde, hvilket var at vise forskellige måder at lave koblinger med modellerne på.\newline{}
API'en er testet i og kan tilgås via Postman.

\section*{Database}
Til projektet er oprettet følgende tabeller:
\textbf{AdvancedSkills}, \textbf{BasicSkills}, \textbf{Characteristics}, \textbf{Characters}, \textbf{ExperienceLogs}, \textbf{Parties} og \textbf{Talents}

\section*{Refleksioner og Videreudvikling}
Projektet er slået for stort op og jeg skulle have begrænset det meget mere tidligere.\newline{}
Endvidere skulle det være mindre \textit{frontend}-afhængig, således det var lettere at vise frem. Jeg håber dog på lige at gøre nogle af frontned tingene færdige til eksamen, således intentionen bedre kan forklares.\newline{}
I forhold til videreudvikling er de åbenlyse ting selvfølgelig er færdigudvikle frontned og implementere de sidste klasser, men udover det, så ville jeg gerne implementere langt flere modeller, så som \textit{careers} og have indtastet mere information, således mere er "automatiseret" for brugeren.

\end{document}