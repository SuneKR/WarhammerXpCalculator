%%%%%%
%	Præamble
%%%%%%

\documentclass{article}

%%%%%%
%	Præambel
%%%%%%

%\documentclass{report}

%	Pakker

\usepackage[utf8]{inputenc}
\usepackage[T1]{fontenc}
\usepackage{graphicx}
\usepackage[cc]{titlepic}
\usepackage{verbatim}
\usepackage{lipsum}
\usepackage{rotating}
\usepackage{fancyvrb}
\usepackage{titling}
\usepackage{listings}
\usepackage{hyperref}
\usepackage[dvipsnames,table]{xcolor}
\usepackage{pdfpages}
\usepackage{float}
\usepackage[danish]{babel}
\usepackage{datetime}
\usepackage{tabularx}
\usepackage{newclude}
\usepackage{tikz}
\usepackage{tablefootnote}

%	Bibloteker

\usetikzlibrary{shapes, arrows}

%	Opsætning

\graphicspath{{./billeder/}}

%	kommandoer

\newcommand{\pic}[2][png]{
	\includegraphics[width=\textwidth]{./#2.#1}
}

\newcommand{\fragCom}[1]{
	\textcolor{LimeGreen}{\texttt{\textbf{#1}}}
}

\newcommand{\logDay}[2][2]{\section*{\formatdate{#2}{#1}{2023}}}

%environmental setup

\lstset{
	breaklines=true,
	breakatwhitespace=true,
	texcl=true,
	extendedchars=false,
	frame=single,
	tabsize=2
}

\lstset{literate=%
	{æ}{{\ae}}1
	{ø}{{\o}}1
	{å}{{\aa}}1
}

%	titel and Aarhus Tech titlecard
\newcommand{\writer}{Sune Koch Rønnow}
%\newcommand{\advisor}{Rasmus Ladefoged Wolffram \& Kenneth Løvgren}
\newcommand{\advisorTwo}{Rasmus Ladefoged Wolffram}
\newcommand{\advisorOne}{Kenneth Løvgren}
\newcommand{\advisor}{\advisorOne{} \& \advisorTwo{}}
\newcommand{\projectName}{Obligentle}
\newcommand{\reportType}{unavngivenreport}
\newcommand{\reportName}{
	\projectName{}: \reportType{}
}

\newcommand{\subtitle}[1]{%
	\posttitle{%
		\par\end{center}
	\begin{center}\large#1\end{center}
	\vskip0.5em}%
}

\title{\reportName{}}
\subtitle{Svendeprøve ved \\ \advisorOne{} \\ \& \\ \advisorTwo{} \\ \vspace{0.75\baselineskip} Aarhus Tech}
\author{\writer{} \\ sune@kochroennow.dk}
\date{\today}
%\date{\formatdate{6}{10}{2023}}

\newcommand{\makeTechTitlecard}{
	\chapter*{Titelblad}
	\begin{table}[h]
	\center
	\begin{tabularx}{\textwidth}{p{.3\linewidth} X}
	\textbf{Deltagere}		&	\writer{}												\\
	\textbf{Projektnavn} 	&	\projectName{}											\\
	\textbf{Skole}			&	Aarhus Tech \newline{} Hasselager Allé 2, 8260 Viby J	    \\
	\textbf{Projektperiode}	&	\formatdate{13}{11}{2023} - \formatdate{15}{12}{2023}		\\
	\textbf{Afleveringsdato}&	\formatdate{8}{12}{2023}									\\
	\textbf{Vejleder}		&	\advisor{}												\\
	\end{tabularx}
	\end{table}
	\section*{Underskrifter}
	\vspace{3\baselineskip}
	\hrule
	\noindent\small \writer{} \null\hfill Dato\\
	\vspace{2\baselineskip}
	\hrule
	\noindent\small \advisorOne{} \null\hfill Dato\\
	\vspace{2\baselineskip}
	\hrule
	\noindent\small \advisorTwo{} \null\hfill Dato\\
}

% tikz setup

\tikzstyle{terminator} = [rectangle, draw, text centered, rounded corners, minimum height=2em, fill=Magenta!40]
\tikzstyle{process} = [rectangle, draw, text centered, minimum height=2em, fill=Blue!40]
\tikzstyle{positive} = [rectangle, draw, text centered, minimum height=2em, fill=Green!40]
\tikzstyle{negative} = [rectangle, draw, text centered, minimum height=2em, fill=Red!40]
\tikzstyle{decision} = [diamond, draw, text centered, minimum height=2em, fill=Yellow!40]
\tikzstyle{input}=[trapezium, draw, text centered, trapezium left angle=60, trapezium right angle=120, minimum height=2em, fill=Cyan!40]
\tikzstyle{connector} = [draw, -latex']
\tikzstyle{semiConnector} = [draw, -latex',dotted]

%%%%%%
%	Indhold
%%%%%%

\begin{document}

\section*{Warhammer Fantasy Roleplay Character Manager}
Rollespil er min store passion og har været det over 20 år, men et stående problemer er, at der kan være en en del papirarbejde at holde styr på.\newline{}
Det er selvfølgelig ikke ens for alle system og et system, hvor mine spillere konstant har bedt mig regne deres karakter efter er \textit{Warhammer Fantasy Roleplay}. I startet med at være fokusere meget på xp udregning, men endte med at blive nedprioriteret til et \textit{MVP}, som er bare en \textit{"character manager"}.

\section*{Produkt}
Produktet er en \textit{Blazor Web App} skrevet i \textit{Core 8.0}.\newline{}
Der er bygget en backend API med 7 modeller, 1 interface og 7 controllers.\newline{}
\textbf{Modellerne} og \textbf{interface}'n er bygger fra bunden\newline{}
\textbf{Controller}'erne er genere\textit{Entity Framework} og justerede og udbygget efter behov.\par{}

På Clientsiden giver \textit{Characters.razor} og \textit{Parties.razor} et overblik over de indtastede medlemmer af henholdsvis \textbf{Character} og \textbf{Party} klasserne.\newline{}
\textit{CharacterSheet.razor} udgør \textbf{edit} funktionen til \textbf{Character}-instanserne og er ikke færdig. Flere af implementeringer afhang dog af den side, så derfor er 4 af \textit{controller}'ne/\textit{modelerne} ikke rigtigt implementeret, men jeg har bibeholdt dem for at man kan se arbejdet der foreligger.\par{}

I forhold til \textit{API} er det nok værd særligt kigge \textbf{Character}-klasse og dens initialisering af \textbf{Characteristic}-klasen, som er gjort via API-kald. I begge \textbf{Skill} klasser arbejder jeg med tilkobling på en anden måde, hvilket var at vise forskellige måder at lave koblinger med modellerne på.\newline{}
API'en er testet i og kan tilgås via Postman.

\section*{Database}
Til projektet er oprettet følgende tabeller:
\textbf{AdvancedSkills}, \textbf{BasicSkills}, \textbf{Characteristics}, \textbf{Characters}, \textbf{ExperienceLogs}, \textbf{Parties} og \textbf{Talents}

\section*{Refleksioner og Videreudvikling}
Projektet er slået for stort op og jeg skulle have begrænset det meget mere tidligere.\newline{}
Endvidere skulle det være mindre \textit{frontend}-afhængig, således det var lettere at vise frem. Jeg håber dog på lige at gøre nogle af frontned tingene færdige til eksamen, således intentionen bedre kan forklares.\newline{}
I forhold til videreudvikling er de åbenlyse ting selvfølgelig er færdigudvikle frontned og implementere de sidste klasser, men udover det, så ville jeg gerne implementere langt flere modeller, så som \textit{careers} og have indtastet mere information, således mere er "automatiseret" for brugeren.

\end{document}